\chapter{外文資料原文}
\label{cha:engorg}

\title{The title of the English paper}

\textbf{Abstract:} As one of the most widely used techniques in operations
research, \emph{ mathematical programming} is defined as a means of maximizing a
quantity known as \emph{bjective function}, subject to a set of constraints
represented by equations and inequalities. Some known subtopics of mathematical
programming are linear programming, nonlinear programming, multiobjective
programming, goal programming, dynamic programming, and multilevel
programming$^{[1]}$.

It is impossible to cover in a single chapter every concept of mathematical
programming. This chapter introduces only the basic concepts and techniques of
mathematical programming such that readers gain an understanding of them
throughout the book$^{[2,3]}$.


\section{Single-Objective Programming}
The general form of single-objective programming (SOP) is written
as follows,
\begin{equation}\tag*{(123)} % 如果附錄中的公式不想讓它出現在公式索引中,那就請
                             % 用 \tag*{xxxx}
\left\{\begin{array}{l}
\max \,\,f(x)\\[0.1 cm]
\mbox{subject to:} \\ [0.1 cm]
\qquad g_j(x)\le 0,\quad j=1,2,\cdots,p
\end{array}\right.
\end{equation}
which maximizes a real-valued function $f$ of
$x=(x_1,x_2,\cdots,x_n)$ subject to a set of constraints.

\newtheorem{mpdef}{Definition}[chapter]
\begin{mpdef}
In SOP, we call $x$ a decision vector, and
$x_1,x_2,\cdots,x_n$ decision variables. The function
$f$ is called the objective function. The set
\begin{equation}\tag*{(456)} % 這裏同理,其它不再一一指定。
S=\left\{x\in\Re^n\bigm|g_j(x)\le 0,\,j=1,2,\cdots,p\right\}
\end{equation}
is called the feasible set. An element $x$ in $S$ is called a
feasible solution.
\end{mpdef}

\newtheorem{mpdefop}[mpdef]{Definition}
\begin{mpdefop}
A feasible solution $x^*$ is called the optimal
solution of SOP if and only if
\begin{equation}
f(x^*)\ge f(x)
\end{equation}
for any feasible solution $x$.
\end{mpdefop}

One of the outstanding contributions to mathematical programming was known as
the Kuhn-Tucker conditions\ref{eq:ktc}. In order to introduce them, let us give
some definitions. An inequality constraint $g_j(x)\le 0$ is said to be active at
a point $x^*$ if $g_j(x^*)=0$. A point $x^*$ satisfying $g_j(x^*)\le 0$ is said
to be regular if the gradient vectors $\nabla g_j(x)$ of all active constraints
are linearly independent.

Let $x^*$ be a regular point of the constraints of SOP and assume that all the
functions $f(x)$ and $g_j(x),j=1,2,\cdots,p$ are differentiable. If $x^*$ is a
local optimal solution, then there exist Lagrange multipliers
$\lambda_j,j=1,2,\cdots,p$ such that the following Kuhn-Tucker conditions hold,
\begin{equation}
\label{eq:ktc}
\left\{\begin{array}{l}
    \nabla f(x^*)-\sum\limits_{j=1}^p\lambda_j\nabla g_j(x^*)=0\\[0.3cm]
    \lambda_jg_j(x^*)=0,\quad j=1,2,\cdots,p\\[0.2cm]
    \lambda_j\ge 0,\quad j=1,2,\cdots,p.
\end{array}\right.
\end{equation}
If all the functions $f(x)$ and $g_j(x),j=1,2,\cdots,p$ are convex and
differentiable, and the point $x^*$ satisfies the Kuhn-Tucker conditions
(\ref{eq:ktc}), then it has been proved that the point $x^*$ is a global optimal
solution of SOP.

\subsection{Linear Programming}
\label{sec:lp}

If the functions $f(x),g_j(x),j=1,2,\cdots,p$ are all linear, then SOP is called
a {\em linear programming}.

The feasible set of linear is always convex. A point $x$ is called an extreme
point of convex set $S$ if $x\in S$ and $x$ cannot be expressed as a convex
combination of two points in $S$. It has been shown that the optimal solution to
linear programming corresponds to an extreme point of its feasible set provided
that the feasible set $S$ is bounded. This fact is the basis of the {\em simplex
  algorithm} which was developed by Dantzig as a very efficient method for
solving linear programming.
\begin{table}[ht]
\centering
  \centering
  \caption*{Table~1\hskip1em This is an example for manually numbered table, which
    would not appear in the list of tables}
  \label{tab:badtabular2}
  \begin{tabular}[c]{|m{1.5cm}|c|c|c|c|c|c|}\hline
    \multicolumn{2}{|c|}{Network Topology} & \# of nodes &
    \multicolumn{3}{c|}{\# of clients} & Server \\\hline
    GT-ITM & Waxman Transit-Stub & 600 &
    \multirow{2}{2em}{2\%}&
    \multirow{2}{2em}{10\%}&
    \multirow{2}{2em}{50\%}&
    \multirow{2}{1.2in}{Max. Connectivity}\\\cline{1-3}
    \multicolumn{2}{|c|}{Inet-2.1} & 6000 & & & &\\\hline
    \multirow{2}{1.5cm}{Xue} & Rui  & Ni &\multicolumn{4}{c|}{\multirow{2}*{}}\\\cline{2-3}
    & \multicolumn{2}{c|}{ABCDEF} &\multicolumn{4}{c|}{} \\\hline
\end{tabular}
\end{table}

Roughly speaking, the simplex algorithm examines only the extreme points of the
feasible set, rather than all feasible points. At first, the simplex algorithm
selects an extreme point as the initial point. The successive extreme point is
selected so as to improve the objective function value. The procedure is
repeated until no improvement in objective function value can be made. The last
extreme point is the optimal solution.

\subsection{Nonlinear Programming}

If at least one of the functions $f(x),g_j(x),j=1,2,\cdots,p$ is nonlinear, then
SOP is called a {\em nonlinear programming}.

A large number of classical optimization methods have been developed to treat
special-structural nonlinear programming based on the mathematical theory
concerned with analyzing the structure of problems.
\begin{figure}[h]
  \centering
  \caption*{Figure~1\quad This is an example for manually numbered figure,
    which would not appear in the list of figures}
  \label{tab:badfigure2}
\end{figure}

Now we consider a nonlinear programming which is confronted solely with
maximizing a real-valued function with domain $\Re^n$.  Whether derivatives are
available or not, the usual strategy is first to select a point in $\Re^n$ which
is thought to be the most likely place where the maximum exists. If there is no
information available on which to base such a selection, a point is chosen at
random. From this first point an attempt is made to construct a sequence of
points, each of which yields an improved objective function value over its
predecessor. The next point to be added to the sequence is chosen by analyzing
the behavior of the function at the previous points. This construction continues
until some termination criterion is met. Methods based upon this strategy are
called {\em ascent methods}, which can be classified as {\em direct methods},
{\em gradient methods}, and {\em Hessian methods} according to the information
about the behavior of objective function $f$. Direct methods require only that
the function can be evaluated at each point. Gradient methods require the
evaluation of first derivatives of $f$. Hessian methods require the evaluation
of second derivatives. In fact, there is no superior method for all
problems. The efficiency of a method is very much dependent upon the objective
function.

\subsection{Integer Programming}

{\em Integer programming} is a special mathematical programming in which all of
the variables are assumed to be only integer values. When there are not only
integer variables but also conventional continuous variables, we call it {\em
  mixed integer programming}. If all the variables are assumed either 0 or 1,
then the problem is termed a {\em zero-one programming}. Although integer
programming can be solved by an {\em exhaustive enumeration} theoretically, it
is impractical to solve realistically sized integer programming problems. The
most successful algorithm so far found to solve integer programming is called
the {\em branch-and-bound enumeration} developed by Balas (1965) and Dakin
(1965). The other technique to integer programming is the {\em cutting plane
  method} developed by Gomory (1959).

\hfill\textit{Uncertain Programming\/}\quad(\textsl{BaoDing Liu, 2006.2})

\section*{References}
\noindent{\itshape NOTE: These references are only for demonstration. They are
  not real citations in the original text.}

\chapter{外文資料的調研閱讀報告或書面翻譯}

\title{英文資料的中文標題}

摘要:本章為外文資料翻譯內容。如果有摘要可以直接寫上來,這部分好像沒有
明確的規定。

\section{單目標規劃}
北冥有魚,其名為鯤。鯤之大,不知其幾千裏也。化而為鳥,其名為鵬。鵬之背,不知其幾
千裏也。怒而飛,其翼若垂天之雲。是鳥也,海運則將徙於南冥。南冥者,天池也。
\begin{equation}\tag*{(123)}
 p(y|\mathbf{x}) = \frac{p(\mathbf{x},y)}{p(\mathbf{x})}=
\frac{p(\mathbf{x}|y)p(y)}{p(\mathbf{x})}
\end{equation}

吾生也有涯,而知也無涯。以有涯隨無涯,殆已!已而為知者,殆而已矣!為善無近名,為
惡無近刑,緣督以為經,可以保身,可以全生,可以養親,可以盡年。

\subsection{線性規劃}
庖丁為文惠君解牛,手之所觸,肩之所倚,足之所履,膝之所倚,砉然響然,奏刀騞然,莫
不中音,合於桑林之舞,乃中經首之會。
\begin{table}[ht]
\centering
  \centering
  \caption*{表~1\hskip1em 這是手動編號但不出現在索引中的一個表格例子}
  \label{tab:badtabular3}
  \begin{tabular}[c]{|m{1.5cm}|c|c|c|c|c|c|}\hline
    \multicolumn{2}{|c|}{Network Topology} & \# of nodes &
    \multicolumn{3}{c|}{\# of clients} & Server \\\hline
    GT-ITM & Waxman Transit-Stub & 600 &
    \multirow{2}{2em}{2\%}&
    \multirow{2}{2em}{10\%}&
    \multirow{2}{2em}{50\%}&
    \multirow{2}{1.2in}{Max. Connectivity}\\\cline{1-3}
    \multicolumn{2}{|c|}{Inet-2.1} & 6000 & & & &\\\hline
    \multirow{2}{1.5cm}{Xue} & Rui  & Ni &\multicolumn{4}{c|}{\multirow{2}*{}}\\\cline{2-3}
    & \multicolumn{2}{c|}{ABCDEF} &\multicolumn{4}{c|}{} \\\hline
\end{tabular}
\end{table}

文惠君曰:“嘻,善哉!技蓋至此乎?”庖丁釋刀對曰:“臣之所好者道也,進乎技矣。始臣之
解牛之時,所見無非全牛者;三年之後,未嘗見全牛也;方今之時,臣以神遇而不以目視,
官知止而神欲行。依乎天理,批大郤,導大窾,因其固然。技經肯綮之未嘗,而況大坬乎!
良庖歲更刀,割也;族庖月更刀,折也;今臣之刀十九年矣,所解數千牛矣,而刀刃若新發
於硎。彼節者有間而刀刃者無厚,以無厚入有間,恢恢乎其於遊刃必有余地矣。是以十九年
而刀刃若新發於硎。雖然,每至於族,吾見其難為,怵然為戒,視為止,行為遲,動刀甚微,
謋然已解,如土委地。提刀而立,為之而四顧,為之躊躇滿志,善刀而藏之。”

文惠君曰:“善哉!吾聞庖丁之言,得養生焉。”


\subsection{非線性規劃}
孔子與柳下季為友,柳下季之弟名曰盜跖。盜跖從卒九千人,橫行天下,侵暴諸侯。穴室樞
戶,驅人牛馬,取人婦女。貪得忘親,不顧父母兄弟,不祭先祖。所過之邑,大國守城,小
國入保,萬民苦之。孔子謂柳下季曰:“夫為人父者,必能詔其子;為人兄者,必能教其弟。
若父不能詔其子,兄不能教其弟,則無貴父子兄弟之親矣。今先生,世之才士也,弟為盜
跖,為天下害,而弗能教也,丘竊為先生羞之。丘請為先生往說之。”
\begin{figure}[h]
  \centering
  \caption*{圖~1\hskip1em 這是手動編號但不出現索引中的圖片的例子}
  \label{tab:badfigure3}
\end{figure}

柳下季曰:“先生言為人父者必能詔其子,為人兄者必能教其弟,若子不聽父之詔,弟不受
兄之教,雖今先生之辯,將奈之何哉?且跖之為人也,心如湧泉,意如飄風,強足以距敵,
辯足以飾非。順其心則喜,逆其心則怒,易辱人以言。先生必無往。”

孔子不聽,顏回為馭,子貢為右,往見盜跖。

\subsection{整數規劃}
盜跖乃方休卒徒大山之陽,膾人肝而餔之。孔子下車而前,見謁者曰:“魯人孔丘,聞將軍
高義,敬再拜謁者。”謁者入通。盜跖聞之大怒,目如明星,發上指冠,曰:“此夫魯國之
巧偽人孔丘非邪?為我告之:爾作言造語,妄稱文、武,冠枝木之冠,帶死牛之脅,多辭繆
說,不耕而食,不織而衣,搖唇鼓舌,擅生是非,以迷天下之主,使天下學士不反其本,妄
作孝弟,而僥幸於封侯富貴者也。子之罪大極重,疾走歸!不然,我將以子肝益晝餔之膳。”


\chapter{其它附錄}
前面兩個附錄主要是給本科生做例子。其它附錄的內容可以放到這裏,當然如果你願意,可
以把這部分也放到獨立的文件中,然後將其到主文件中。
