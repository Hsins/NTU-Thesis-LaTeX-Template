\chapter{帶 English 的標題}
\label{cha:intro}

這是的示例文檔,基本上覆蓋了模板中所有格式的設置。建議大家在使用模板之前,除了閱讀《用戶手冊》,這個示例文檔也最好能看一看。

小老鼠偷吃熱涼粉;短長蟲環繞矮高粱\footnote{韓愈(768-824),字退之,河南河陽(今河南孟縣)人,自稱郡望昌黎,世稱韓昌黎。幼孤貧刻苦好學,德宗貞元八年進士。曾任監察禦史,因上疏請免關中賦役,貶為陽山縣令。後隨宰相裴度平定淮西遷刑部侍郎,又因上表諫迎佛骨,貶潮州刺史。做過吏部侍郎,死謚文公,故世稱韓吏部、韓文公。是唐代古文運動領袖,與柳宗元合稱韓柳。詩力求險怪新奇,雄渾重氣勢。}。


\section{封面相關}
封面的例子請參看 \texttt{cover.tex}。主要符號表參看 \texttt{denation.tex},附錄和個人簡歷分別參看 \texttt{appendix01.tex} 和 \texttt{resume.tex}。裏面的命令都很直觀,一看即會\footnote{你說還是看不懂?怎麽會呢?}。

\section{字體命令}
\label{sec:first}

蘇軾(1037-1101),北宋文學家、書畫家。字子瞻,號東坡居士,眉州眉山(今屬四川)人。蘇洵子。嘉佑進士。神宗時曾任祠部員外郎,因反對王安石新法而求外職,任杭州通判,知密州、徐州、湖州。後以作詩“謗訕朝廷”罪貶黃州。哲宗時任翰林學士,曾出知杭州、穎州等,官至禮部尚書。後又貶謫惠州、儋州。北還後第二年病死常州。南宋時追謚文忠。與父洵弟轍,合稱“三蘇”。在政治上屬於舊黨,但也有改革弊政的要求。其文汪洋恣肆,明白暢達,為“唐宋八大家”之一。  其詩清新豪健,善用誇張比喻,在藝術表現方面獨具風格。少數詩篇也能反映民間疾苦,指責統治者的奢侈驕縱。詞開豪放一派,對後代很有影響。《念奴嬌·赤壁懷古》、《水調歌頭·丙辰中秋》傳誦甚廣。

坡仙擅長行書、楷書,取法李邕、徐浩、顏真卿、楊凝式,而能自創新意。用筆豐腴跌宕,有天真爛漫之趣。與蔡襄、黃庭堅、米芾並稱“宋四家”。能畫竹,學文同,也喜作枯木怪石。論畫主張“神似”,認為“論畫以形似,見與兒童鄰”;高度評價“詩中有畫,畫中有詩”的藝術造詣。詩文有《東坡七集》等。存世書跡有《答謝民師論文帖》、《祭黃幾道文》、《前赤壁賦》、《黃州寒食詩帖》等。畫跡有《枯木怪石圖》、《  竹石圖》等。

易與天地準,故能彌綸天地之道。仰以觀於天文,俯以察於地理,是故知幽明之故。原始反終,故知死生之說。精氣為物,遊魂為變,是故知鬼神之情狀。與天地相似,故不違。知周乎萬物,而道濟天下,故不過。旁行而不流,樂天知命,故不憂。安土敦乎仁,故能愛。範圍天地之化而不過,曲成萬物而不遺,通乎晝夜之道而知,故神無方而易無體。

有天地,然後萬物生焉。盈天地之間者,唯萬物,故受之以屯;屯者盈也,屯者物之始生也。物生必蒙,故受之以蒙;蒙者蒙也,物之穉也。物穉不可不養也,故受之以需;需者飲食之道也。飲  食必有訟,故受之以訟。訟必有眾起,故受之以師;師者眾也。眾必有所比,故受之以比;  比者比也。比必有所畜也,故受之以小畜。物畜然後有禮,故受之以履。

履而泰,然後安,故受之以泰;泰者通也。物不可以終通,故受之以否。物不可以終否,故受之以同人。與人同者,物必歸焉,故受之以大有。有大者不可以盈,故受之以謙。有大而能謙,必豫,故受之以豫。豫必有隨,故受之以隨。以喜隨人者,必有事,故受之以蠱;蠱者事也有事而後可大,故受
之以臨;臨者大也。物大然後可觀,故受之以觀。可觀而後有所合,故受之以噬嗑;嗑者合也。物不可以茍合而已,故受之以賁;賁者飾也。致飾然後亨,則盡矣,故受之以剝;者剝也。物不可以終盡,剝窮上反下,故受之以覆。覆則不妄矣,故受之以無妄。

有無妄然後可畜,故受之以大畜。物畜然後可養,故受之以頤;頤者養也。不養則不動,故受之以大過。物不可以終過,故受之以坎;坎者陷也。陷必有所麗,故受之以離;離者麗也。

\section{表格樣本}
\label{chap1:sample:table} 

\subsection{基本表格}
\label{sec:basictable}

模板中關於表格的宏包有三個:命令有一個。三線表可以用能很好的配合使用。如果表格比較簡單的話可以直接用命令控制。

\begin{table}[htb]
  \centering
如果想在表格中使用腳註,minipage是個不錯的辦法
    \begin{tabularx}{\linewidth}{lX}
      \toprule[1.5pt]
      文件名 & 描述 \\\midrule[1pt]
      thuthesis.ins & \LaTeX{} 安裝文件,\textsc{DocStrip}\footnote{表格中的腳註} \\
      thuthesis.dtx & 所有的一切都在這裏面\footnote{再來一個}。\\
      thuthesis.cls & 模板類文件。\\
      thuthesis.cfg & 模板配置文。cls 和 cfg 由前兩個文件生成。\\
      thuthesis-numeric.bst    & 參考文獻 BIB\TeX\ 樣式文件。\\
      thuthesis-author-year.bst    & 參考文獻 BIB\TeX\ 樣式文件。\\
      thuthesis.sty   & 常用的包和命令寫在這裏,減輕主文件的負擔。\\
      \bottomrule[1.5pt]
    \end{tabularx}
\end{table}

首先來看一個最簡單的表格。表列舉了本模板主要文件及其功
能。請大家註意三線表中各條線對應的命令。這個例子還展示了如何在表格中正確使用腳註。
由於本身不支持在表格中使用,所以我們不得不將表格放在
小頁中,而且最好將表格的寬度設置為小頁的寬度,這樣腳註看起來才更美觀。

\subsection{覆雜表格}
\label{sec:complicatedtable}

我們經常會在表格下方標註數據來源,或者對表格裏面的條目進行解釋。前面的腳註是一種
不錯的方法,如果不喜歡腳註,可以在表格後面寫註釋,比如表~\ref{tab:tabexamp1}。
\begin{table}[htbp]
  \centering
  \caption{覆雜表格示例 1。這個引用不會導致編號混亂。}
  \label{tab:tabexamp1}
  \begin{minipage}[t]{0.8\textwidth} 
    \begin{tabularx}{\linewidth}{|l|X|X|X|X|}
      \hline
      \multirow{2}*{\diagbox[width=5em]{x}{y}} & \multicolumn{2}{c|}{First Half} & \multicolumn{2}{c|}{Second Half}\\\cline{2-5}
      & 1st Qtr &2nd Qtr&3rd Qtr&4th Qtr \\ \hline
      East$^{*}$ &   20.4&   27.4&   90&     20.4 \\
      West$^{**}$ &   30.6 &   38.6 &   34.6 &  31.6 \\ \hline
    \end{tabularx}\\[2pt]
    \footnotesize 註:數據來源《使用手冊》。\\
    *:東部\\
    **:西部
  \end{minipage}
\end{table}

此外,表同時還演示了另外兩個功能:1)通過的
 \texttt{|X|} 擴展實現表格自動放大;2)通過命令在表頭部分
插入反斜線。

為了使我們的例子更接近實際情況,我會在必要的時候插入一些“無關”文字,以免太多圖
表同時出現,導致排版效果不太理想。第一個出場的當然是我的最愛:風流瀟灑、駿馬絕塵、
健筆淩雲的李太白了。

李白,字太白,隴西成紀人。涼武昭王暠九世孫。或曰山東人,或曰蜀人。白少有逸才,志
氣宏放,飄然有超世之心。初隱岷山,益州長史蘇颋見而異之,曰:“是子天才英特,可比
相如。”天寶初,至長安,往見賀知章。知章見其文,嘆曰:“子謫仙人也。”言於明皇,
召見金鑾殿,奏頌一篇。帝賜食,親為調羹,有詔供奉翰林。白猶與酒徒飲於市,帝坐沈香
亭子,意有所感,欲得白為樂章,召入,而白已醉。左右以水颒面,稍解,援筆成文,婉麗
精切。帝愛其才,數宴見。白常侍帝,醉,使高力士脫靴。力士素貴,恥之,摘其詩以激楊
貴妃。帝欲官白,妃輒沮止。白自知不為親近所容,懇求還山。帝賜金放還。乃浪跡江湖,
終日沈飲。永王璘都督江陵,辟為僚佐。璘謀亂,兵敗,白坐長流夜郎,會赦得還。族人陽
冰為當塗令,白往依之。代宗立,以左拾遺召,而白已卒。文宗時,詔以白歌詩、裴旻劍舞、
張旭草書為三絕雲。集三十卷。今編詩二十五卷。\hfill —— 《全唐詩》詩人小傳

\begin{table}[htbp]
\noindent\begin{minipage}{0.5\textwidth}
\centering
\caption{第一個並排子表格}
\label{tab:parallel1}
\begin{tabular}{p{2cm}p{2cm}}
\toprule[1.5pt]
111 & 222 \\\midrule[1pt]
222 & 333 \\\bottomrule[1.5pt]
\end{tabular}
\end{minipage}%
\begin{minipage}{0.5\textwidth}
\centering
\caption{第二個並排子表格}
\label{tab:parallel2}
\begin{tabular}{p{2cm}p{2cm}}
\toprule[1.5pt]
111 & 222 \\\midrule[1pt]
222 & 333 \\\bottomrule[1.5pt]
\end{tabular}
\end{minipage}
\end{table}

然後就是憂國憂民,詩家楷模杜工部了。杜甫,字子美,其先襄陽人,曾祖依藝為鞏令,因
居鞏。甫天寶初應進士,不第。後獻《三大禮賦》,明皇奇之,召試文章,授京兆府兵曹參
軍。安祿山陷京師,肅宗即位靈武,甫自賊中遁赴行在,拜左拾遺。以論救房琯,出為華州
司功參軍。關輔饑亂,寓居同州同谷縣,身自負薪采梠,餔糒不給。久之,召補京兆府功曹,
道阻不赴。嚴武鎮成都,奏為參謀、檢校工部員外郎,賜緋。武與甫世舊,待遇甚厚。乃於
成都浣花裏種竹植樹,枕江結廬,縱酒嘯歌其中。武卒,甫無所依,乃之東蜀就高適。既至
而適卒。是歲,蜀帥相攻殺,蜀大擾。甫攜家避亂荊楚,扁舟下峽,未維舟而江陵亦亂。乃
溯沿湘流,遊衡山,寓居耒陽。卒年五十九。元和中,歸葬偃師首陽山,元稹志其墓。天寶
間,甫與李白齊名,時稱李杜。然元稹之言曰:“李白壯浪縱恣,擺去拘束,誠亦差肩子美
矣。至若鋪陳終始,排比聲韻,大或千言,次猶數百,詞氣豪邁,而風調清深,屬對律切,
而脫棄凡近,則李尚不能歷其藩翰,況堂奧乎。”白居易亦雲:“杜詩貫穿古今,  盡工盡
善,殆過於李。”元、白之論如此。蓋其出處勞佚,喜樂悲憤,好賢惡惡,一見之於詩。而
又以忠君憂國、傷時念亂為本旨。讀其詩可以知其世,故當時謂之“詩史”。舊集詩文共六
十卷,今編詩十九卷。

\begin{table}[htbp]
\centering
\caption{並排子表格}
\label{tab:subtable}
\subcaptionbox{第一個子表格}
{
\begin{tabular}{p{2cm}p{2cm}}
\toprule[1.5pt]
111 & 222 \\\midrule[1pt]
222 & 333 \\\bottomrule[1.5pt]
\end{tabular}
}
\hskip2cm
\subcaptionbox{第二個子表格}
{
\begin{tabular}{p{2cm}p{2cm}}
\toprule[1.5pt]
111 & 222 \\\midrule[1pt]
222 & 333 \\\bottomrule[1.5pt]
\end{tabular}
}
\end{table}

不可否認 \LaTeX{} 的表格功能沒有想象中的那麽強大,不過只要足夠認真,足夠細致,
同樣可以排出來非常覆雜非常漂亮的表格。請參看表。
\begin{table}[htbp]
  \centering
  \caption{覆雜表格示例 2}
  \label{tab:tabexamp2}
  \begin{tabular}[c]{|m{1.5cm}|c|c|c|c|c|c|}\hline
    \multicolumn{2}{|c|}{Network Topology} & \# of nodes & 
    \multicolumn{3}{c|}{\# of clients} & Server \\\hline
    GT-ITM & Waxman Transit-Stub & 600 &
    \multirow{2}{1em}{2\%}&
    \multirow{2}{1.5em}{10\%}&
    \multirow{2}{1.5em}{50\%}&
    \multirow{2}{1.2in}{Max. Connectivity}\\\cline{1-3}
    \multicolumn{2}{|c|}{Inet-2.1} & 6000 & & & &\\\hline
    \multirow{2}{1.5cm}{Xue} & Rui  & Ni &\multicolumn{4}{c|}{\multirow{2}*{}\\\cline{2-3}
    & \multicolumn{2}{c|}{ABCDEF} &\multicolumn{4}{c|}{} \\\hline
\end{tabular}
\end{table}

最後就是清新飄逸、文約意賅、空谷絕響的王大俠了。王維,字摩詰,河東人。工書畫,與
弟縉俱有俊才。開元九年,進士擢第,調太樂丞。坐累為濟州司倉參軍,歷右拾遺、監察禦
史、左補闕、庫部郎中,拜吏部郎中。天寶末,為給事中。安祿山陷兩都,維為賊所得,服
藥陽喑,拘於菩提寺。祿山宴凝碧池,維潛賦詩悲悼,聞於行在。賊平,陷賊官三等定罪,
特原之,責授太子中允,遷中庶子、中書舍人。覆拜給事中,轉尚書右丞。維以詩名盛於開
元、天寶間,寧薛諸王駙馬豪貴之門,無不拂席迎之。得宋之問輞川別墅,山水絕勝,與道
友裴迪,浮舟往來,彈琴賦詩,嘯詠終日。篤於奉佛,晚年長齋禪誦。一日,忽索筆作書
數紙,別弟縉及平生親故,舍筆而卒。贈秘書監。寶應中,代宗問縉:“朕常於諸王坐聞維
樂章,今存幾何?”縉集詩六卷,文四卷,表上之。敕答雲,卿伯氏位列先朝,名高希代。
抗行周雅,長揖楚辭。詩家者流,時論歸美。克成編錄,嘆息良深。殷璠謂維詩詞秀調雅,
意新理愜。在泉成珠,著壁成繪。蘇軾亦雲:“維詩中有畫,畫中有詩也。”今編詩四卷。

要想用好論文模板還是得提前學習一些 \TeX/\LaTeX{}的相關知識,具備一些基本能力,掌
握一些常見技巧,否則一旦遇到問題還真是比較麻煩。我們見過很多這樣的同學,一直以來
都是使用 Word 等字處理工具,以為 \LaTeX{}模板的用法也應該類似,所以就沿襲同樣的思
路來對待這種所見非所得的排版工具,結果被折騰的焦頭爛額,疲憊不堪。

如果您要排版的表格長度超過一頁,那麽推薦使用 
\begin{longtable}[c]{c*{6}{r}}
\caption{實驗數據}\label{tab:performance}\\
\toprule[1.5pt]
 測試程序 & \multicolumn{1}{c}{正常運行} & \multicolumn{1}{c}{同步} & \multicolumn{1}{c}{檢查點} & \multicolumn{1}{c}{卷回恢覆}
& \multicolumn{1}{c}{進程遷移} & \multicolumn{1}{c}{檢查點} \\
& \multicolumn{1}{c}{時間 (s)}& \multicolumn{1}{c}{時間 (s)}&
\multicolumn{1}{c}{時間 (s)}& \multicolumn{1}{c}{時間 (s)}& \multicolumn{1}{c}{
  時間 (s)}&  文件(KB)\\\midrule[1pt]
\endfirsthead
\multicolumn{7}{c}{續表~\thetable\hskip1em 實驗數據}\\
\toprule[1.5pt]
 測試程序 & \multicolumn{1}{c}{正常運行} & \multicolumn{1}{c}{同步} & \multicolumn{1}{c}{檢查點} & \multicolumn{1}{c}{卷回恢覆}
& \multicolumn{1}{c}{進程遷移} & \multicolumn{1}{c}{檢查點} \\
& \multicolumn{1}{c}{時間 (s)}& \multicolumn{1}{c}{時間 (s)}&
\multicolumn{1}{c}{時間 (s)}& \multicolumn{1}{c}{時間 (s)}& \multicolumn{1}{c}{
  時間 (s)}&  文件(KB)\\\midrule[1pt]
\endhead
\hline
\multicolumn{7}{r}{續下頁}
\endfoot
\endlastfoot
CG.A.2 & 23.05 & 0.002 & 0.116 & 0.035 & 0.589 & 32491 \\
CG.A.4 & 15.06 & 0.003 & 0.067 & 0.021 & 0.351 & 18211 \\
CG.A.8 & 13.38 & 0.004 & 0.072 & 0.023 & 0.210 & 9890 \\
CG.B.2 & 867.45 & 0.002 & 0.864 & 0.232 & 3.256 & 228562 \\
CG.B.4 & 501.61 & 0.003 & 0.438 & 0.136 & 2.075 & 123862 \\
CG.B.8 & 384.65 & 0.004 & 0.457 & 0.108 & 1.235 & 63777 \\
MG.A.2 & 112.27 & 0.002 & 0.846 & 0.237 & 3.930 & 236473 \\
MG.A.4 & 59.84 & 0.003 & 0.442 & 0.128 & 2.070 & 123875 \\
MG.A.8 & 31.38 & 0.003 & 0.476 & 0.114 & 1.041 & 60627 \\
MG.B.2 & 526.28 & 0.002 & 0.821 & 0.238 & 4.176 & 236635 \\
MG.B.4 & 280.11 & 0.003 & 0.432 & 0.130 & 1.706 & 123793 \\
MG.B.8 & 148.29 & 0.003 & 0.442 & 0.116 & 0.893 & 60600 \\
LU.A.2 & 2116.54 & 0.002 & 0.110 & 0.030 & 0.532 & 28754 \\
LU.A.4 & 1102.50 & 0.002 & 0.069 & 0.017 & 0.255 & 14915 \\
LU.A.8 & 574.47 & 0.003 & 0.067 & 0.016 & 0.192 & 8655 \\
LU.B.2 & 9712.87 & 0.002 & 0.357 & 0.104 & 1.734 & 101975 \\
LU.B.4 & 4757.80 & 0.003 & 0.190 & 0.056 & 0.808 & 53522 \\
LU.B.8 & 2444.05 & 0.004 & 0.222 & 0.057 & 0.548 & 30134 \\
EP.A.2 & 123.81 & 0.002 & 0.010 & 0.003 & 0.074 & 1834 \\
EP.A.4 & 61.92 & 0.003 & 0.011 & 0.004 & 0.073 & 1743 \\
EP.A.8 & 31.06 & 0.004 & 0.017 & 0.005 & 0.073 & 1661 \\
EP.B.2 & 495.49 & 0.001 & 0.009 & 0.003 & 0.196 & 2011 \\
EP.B.4 & 247.69 & 0.002 & 0.012 & 0.004 & 0.122 & 1663 \\
EP.B.8 & 126.74 & 0.003 & 0.017 & 0.005 & 0.083 & 1656 \\
\bottomrule[1.5pt]
\end{longtable}

\subsection{其它}
\label{sec:tableother}
如果不想讓某個表格或者圖片出現在索引裏面,請使用命令 \cs{caption*}。
這個命令不會給表格編號,也就是出來的只有標題文字而沒有“表~XX”,“圖~XX”,否則
索引裏面序號不連續就顯得不倫不類,這也是 \LaTeX{} 裏星號命令默認的規則。

有這種需求的多是本科同學的英文資料翻譯部分,如果覺得附錄中英文原文中的表格和圖
片顯示成“表”和“圖”不協調的話,一個很好的辦法就是用 \cs{caption*},參數
隨便自己寫,比如不守規矩的表~1.111 和圖~1.111 能滿足這種特殊需要(可以參看附錄部
分)。
\begin{table}[ht]
  \begin{minipage}{0.4\linewidth}
    \centering
    \caption*{表~1.111\quad 這是一個手動編號,不出現在索引中的表格。}
    \label{tab:badtabular}
      \framebox(150,50)[c]{\thuthesis}
  \end{minipage}%
  \hfill%
  \begin{minipage}{0.4\linewidth}
    \centering
    \caption*{Figure~1.111\quad 這是一個手動編號,不出現在索引中的圖。}
    \label{tab:badfigure}
    \framebox(150,50)[c]{薛瑞尼}
  \end{minipage}
\end{table}

如果的確想讓它編號,但又不想讓它出現在索引中的話,目前模板上不支持。

最後,雖然大家不一定會獨立使用小頁,但是關於小頁中的腳註還是有必要提一下。請看下
面的例子。

\begin{minipage}[t]{\linewidth-2\parindent}
  柳宗元,字子厚(773-819),河東(今永濟縣)人\footnote{山西永濟水餃。},是唐代
  傑出的文學家,哲學家,同時也是一位政治改革家。與韓愈共同倡導唐代古文運動,並稱
  韓柳\footnote{唐宋八大家之首二位。}。
\end{minipage}

唐朝安史之亂後,宦官專權,藩鎮割據,土地兼並日漸嚴重,社會生產破壞嚴重,民不聊生。柳宗
元對這種社會現實極為不滿,他積極參加了王叔文領導的“永濟革新”,並成為這一
運動的中堅人物。他們革除弊政,打擊權奸,觸犯了宦官和官僚貴族利益,在他們的聯合反
撲下,改革失敗了,柳宗元被貶為永州司馬。

\section{定理環境}
\label{sec:theorem}

給大家演示一下各種和證明有關的環境:

\begin{assumption}
待月西廂下,迎風戶半開;隔墻花影動,疑是玉人來。
\begin{eqnarray}
  \label{eq:eqnxmp}
  c & = & a^2 - b^2\\
    & = & (a+b)(a-b)
\end{eqnarray}
\end{assumption}

千辛萬苦,歷盡艱難,得有今日。然相從數千裏,未曾哀戚。今將渡江,方圖百年歡笑,如
何反起悲傷?(引自《杜十娘怒沈百寶箱》)

\begin{definition}
子曰:「道千乘之國,敬事而信,節用而愛人,使民以時。」
\end{definition}

千古第一定義!問世間、情為何物,只教生死相許?天南地北雙飛客,老翅幾回寒暑。歡樂趣,離別苦,就中更有癡兒女。
君應有語,渺萬裏層雲,千山暮雪,只影向誰去?

橫汾路,寂寞當年簫鼓,荒煙依舊平楚。招魂楚些何嗟及,山鬼暗諦風雨。天也妒,未信與,鶯兒燕子俱黃土。
千秋萬古,為留待騷人,狂歌痛飲,來訪雁丘處。

\begin{proposition}
 曾子曰:「吾日三省吾身 —— 為人謀而不忠乎?與朋友交而不信乎?傳不習乎?」
\end{proposition}

多麽淒美的命題啊!其日牛馬嘶,新婦入青廬,奄奄黃昏後,寂寂人定初,我命絕今日,
魂去屍長留,攬裙脫絲履,舉身赴清池,府吏聞此事,心知長別離,徘徊庭樹下,自掛東南
枝。

\begin{remark}
天不言自高,水不言自流。
\begin{gather*}
\begin{split} 
\varphi(x,z)
&=z-\gamma_{10}x-\gamma_{mn}x^mz^n\\
&=z-Mr^{-1}x-Mr^{-(m+n)}x^mz^n
\end{split}\\[6pt]
\begin{align} \zeta^0&=(\xi^0)^2,\\
\zeta^1 &=\xi^0\xi^1,\\
\zeta^2 &=(\xi^1)^2,
\end{align}
\end{gather*}
\end{remark}

天尊地卑,乾坤定矣。卑高以陳,貴賤位矣。 動靜有常,剛柔斷矣。方以類聚,物以群分,
吉兇生矣。在天成象,在地成形,變化見矣。鼓之以雷霆,潤之以風雨,日月運行,一寒一
暑,乾道成男,坤道成女。乾知大始,坤作成物。乾以易知,坤以簡能。易則易知,簡則易
從。易知則有親,易從則有功。有親則可久,有功則可大。可久則賢人之德,可大則賢人之
業。易簡,而天下矣之理矣;天下之理得,而成位乎其中矣。

\begin{axiom}
兩點間直線段距離最短。  
\begin{align}
x&\equiv y+1\pmod{m^2}\\
x&\equiv y+1\mod{m^2}\\
x&\equiv y+1\pod{m^2}
\end{align}
\end{axiom}

《彖曰》:大哉乾元,萬物資始,乃統天。雲行雨施,品物流形。大明始終,六位時成,時
乘六龍以禦天。乾道變化,各正性命,保合大和,乃利貞。首出庶物,萬國鹹寧。

《象曰》:天行健,君子以自強不息。潛龍勿用,陽在下也。見龍再田,德施普也。終日乾
乾,反覆道也。或躍在淵,進無咎也。飛龍在天,大人造也。亢龍有悔,盈不可久也。用九,
天德不可為首也。   

\begin{lemma}
《貓和老鼠》是我最愛看的動畫片。
\begin{multline*}%\tag*{[a]} % 這個不出現在索引中
\int_a^b\biggl\{\int_a^b[f(x)^2g(y)^2+f(y)^2g(x)^2]
 -2f(x)g(x)f(y)g(y)\,dx\biggr\}\,dy \\
 =\int_a^b\biggl\{g(y)^2\int_a^bf^2+f(y)^2
  \int_a^b g^2-2f(y)g(y)\int_a^b fg\biggr\}\,dy
\end{multline*}
\end{lemma}

行行重行行,與君生別離。相去萬余裏,各在天一涯。道路阻且長,會面安可知。胡馬依北
風,越鳥巢南枝。相去日已遠,衣帶日已緩。浮雲蔽白日,遊子不顧返。思君令人老,歲月
忽已晚。  棄捐勿覆道,努力加餐飯。

\begin{theorem}\label{the:theorem1}
犯我強漢者,雖遠必誅\hfill —— 陳湯(漢)
\end{theorem}
\begin{subequations}
\begin{align}
y & = 1 \\
y & = 0
\end{align}
\end{subequations}
道可道,非常道。名可名,非常名。無名天地之始;有名萬物之母。故常無,欲以觀其妙;
常有,欲以觀其僥。此兩者,同出而異名,同謂之玄。玄之又玄,眾妙之門。上善若水。水
善利萬物而不爭,處眾人之所惡,故幾於道。曲則全,枉則直,窪則盈,敝則新,少則多,
多則惑。人法地,地法天,天法道,道法自然。知人者智,自知者明。勝人者有力,自勝
者強。知足者富。強行者有志。不失其所者久。死而不亡者壽。

\begin{proof}
燕趙古稱多感慨悲歌之士。董生舉進士,連不得志於有司,懷抱利器,郁郁適茲土,吾
知其必有合也。董生勉乎哉?

夫以子之不遇時,茍慕義強仁者,皆愛惜焉,矧燕、趙之士出乎其性者哉!然吾嘗聞
風俗與化移易,吾惡知其今不異於古所雲邪?聊以吾子之行卜之也。董生勉乎哉?

吾因子有所感矣。為我吊望諸君之墓,而觀於其市,覆有昔時屠狗者乎?為我謝
曰:“明天子在上,可以出而仕矣!” \hfill —— 韓愈《送董邵南序》
\end{proof}

\begin{corollary}
  四川話配音的《貓和老鼠》是世界上最好看最好聽最有趣的動畫片。
\begin{alignat}{3}
V_i & =v_i - q_i v_j, & \qquad X_i & = x_i - q_i x_j,
 & \qquad U_i & = u_i,
 \qquad \text{for $i\ne j$;}\label{eq:B}\\
V_j & = v_j, & \qquad X_j & = x_j,
  & \qquad U_j & u_j + \sum_{i\ne j} q_i u_i.
\end{alignat}
\end{corollary}

迢迢牽牛星,皎皎河漢女。
纖纖擢素手,劄劄弄機杼。
終日不成章,泣涕零如雨。
河漢清且淺,相去覆幾許。
盈盈一水間,脈脈不得語。

\begin{example}
  大家來看這個例子。
\begin{equation}
\label{ktc}
\left\{\begin{array}{l}
\nabla f({\mbox{\boldmath $x$}}^*)-\sum\limits_{j=1}^p\lambda_j\nabla g_j({\mbox{\boldmath $x$}}^*)=0\\[0.3cm]
\lambda_jg_j({\mbox{\boldmath $x$}}^*)=0,\quad j=1,2,\cdots,p\\[0.2cm]
\lambda_j\ge 0,\quad j=1,2,\cdots,p.
\end{array}\right.
\end{equation}
\end{example}

\begin{exercise}
  請列出 Andrew S. Tanenbaum 和 W. Richard Stevens 的所有著作。
\end{exercise}

\begin{conjecture} \textit{Poincare Conjecture} If in a closed three-dimensional
  space, any closed curves can shrink to a point continuously, this space can be
  deformed to a sphere.
\end{conjecture}

\begin{problem}
 回答還是不回答,是個問題。 
\end{problem}

如何引用定理~\ref{the:theorem1} 呢?加上 \cs{label} 使用 \cs{ref} 即可。妾發
初覆額,折花門前劇。郎騎竹馬來,繞床弄青梅。同居長幹裏,兩小無嫌猜。 十四為君婦,
羞顏未嘗開。低頭向暗壁,千喚不一回。十五始展眉,願同塵與灰。常存抱柱信,豈上望夫
台。 十六君遠行,瞿塘灩滪堆。五月不可觸,猿聲天上哀。門前遲行跡,一一生綠苔。苔深
不能掃,落葉秋風早。八月蝴蝶來,雙飛西園草。感此傷妾心,坐愁紅顏老。

\section{參考文獻}
\label{sec:bib}
當然參考文獻可以直接寫 \cs{bibitem},雖然費點功夫,但是好控制,各種格式可以自己隨意改
寫。

本模板推薦使用 BIB\TeX,分別提供數字引用(\texttt{thuthesis-numeric.bst})和作
者年份引用(\texttt{thuthesis-author-year.bst})樣式,基本符合學校的參考文獻格式
(如專利等引用未加詳細測試)。看看這個例子,關於書的~\cite{tex, companion,
  ColdSources},還有這些~\cite{Krasnogor2004e, clzs, zjsw},關於雜志
的~\cite{ELIDRISSI94, MELLINGER96, SHELL02},碩士論文~\cite{zhubajie,
  metamori2004},博士論文~\cite{shaheshang, FistSystem01},標準文
件~\cite{IEEE-1363},會議論文~\cite{DPMG,kocher99},技術報告~\cite{NPB2},電子文
獻~\cite{chuban2001,oclc2000}。中文參考文獻~\cite{cnarticle}應增
加 \texttt{language=``chinese''} 字段,以便進行相應處理。另外,本模板對中文文
獻~\cite{cnproceed}的支持並不是十全十美,如果有不如意的地方,請手動修
改 \texttt{bbl} 文件。

有時候不想要上標,那麽可以這樣~\inlinecite{shaheshang},這個非常重要。

有時候一些參考文獻沒有紙質出處,需要標註 URL。缺省情況下,URL 不會在連字符處斷行,
這可能使得用連字符代替空格的網址分行很難看。如果需要,可以將模板類文件中
\begin{verbatim}
\RequirePackage{hyperref}
\end{verbatim}
一行改為:
\begin{verbatim}
\PassOptionsToPackage{hyphens}{url}
\RequirePackage{hyperref}
\end{verbatim}
使得連字符處可以斷行。更多設置可以參考 \texttt{url} 宏包文檔。

\section{公式}
\label{sec:equation}
貝葉斯公式如式~(\ref{equ:chap1:bayes}),其中 $p(y|\mathbf{x})$ 為後驗;
$p(\mathbf{x})$ 為先驗;分母 $p(\mathbf{x})$ 為歸一化因子。
\begin{equation}
\label{equ:chap1:bayes}
p(y|\mathbf{x}) = \frac{p(\mathbf{x},y)}{p(\mathbf{x})}=
\frac{p(\mathbf{x}|y)p(y)}{p(\mathbf{x})} 
\end{equation}

論文裏面公式越多,\TeX{} 就越 happy。再看一個 \pkg{amsmath} 的例子:
\newcommand{\envert}[1]{\left\lvert#1\right\rvert} 
\begin{equation}\label{detK2}
\det\mathbf{K}(t=1,t_1,\dots,t_n)=\sum_{I\in\mathbf{n}}(-1)^{\envert{I}}
\prod_{i\in I}t_i\prod_{j\in I}(D_j+\lambda_jt_j)\det\mathbf{A}
^{(\lambda)}(\overline{I}|\overline{I})=0.
\end{equation} 

前面定理示例部分列舉了很多公式環境,可以說把常見的情況都覆蓋了,大家在寫公式的時
候一定要好好看 \pkg{amsmath} 的文檔,並參考模板中的用法:
\begin{multline*}%\tag{[b]} % 這個出現在索引中的
\int_a^b\biggl\{\int_a^b[f(x)^2g(y)^2+f(y)^2g(x)^2]
 -2f(x)g(x)f(y)g(y)\,dx\biggr\}\,dy \\
 =\int_a^b\biggl\{g(y)^2\int_a^bf^2+f(y)^2
  \int_a^b g^2-2f(y)g(y)\int_a^b fg\biggr\}\,dy
\end{multline*}

其實還可以看看這個多級規劃:
\begin{equation}\label{bilevel}
\left\{\begin{array}{l}
\max\limits_{{\mbox{\footnotesize\boldmath $x$}}} F(x,y_1^*,y_2^*,\cdots,y_m^*)\\[0.2cm]
\mbox{subject to:}\\[0.1cm]
\qquad G(x)\le 0\\[0.1cm]
\qquad(y_1^*,y_2^*,\cdots,y_m^*)\mbox{ solves problems }(i=1,2,\cdots,m)\\[0.1cm]
\qquad\left\{\begin{array}{l}
    \max\limits_{{\mbox{\footnotesize\boldmath $y_i$}}}f_i(x,y_1,y_2,\cdots,y_m)\\[0.2cm]
    \mbox{subject to:}\\[0.1cm]
    \qquad g_i(x,y_1,y_2,\cdots,y_m)\le 0.
    \end{array}\right.
\end{array}\right.
\end{equation}
這些跟規劃相關的公式都來自於劉寶碇老師《不確定規劃》的課件。
